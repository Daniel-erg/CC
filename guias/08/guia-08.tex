\documentclass[11pt]{exam}
\usepackage[activeacute,spanish]{babel} % Permite el idioma espa\~nol.
\usepackage[utf8]{inputenc}
\usepackage{amsmath,epsfig}
\usepackage[colorlinks]{hyperref}

\usepackage{minted} 
\usemintedstyle{emacs}
\usepackage{tcolorbox} % colores para el fondo
\definecolor{bg}{rgb}{0.95,0.95,0.95} %color de fondo

\pagestyle{headandfoot}
\spanishdecimal{.}

\begin{document}

\firstpageheadrule
%\firstpagefootrule
%\firstpagefooter{}{Pagina \thepage\ de \pages}{}
\runningheadrule
%\runningfootrule
\lhead{\bf\normalsize Computaci\'on Cient\'ifica\\ Gu\'ia 08}
\rhead{\bf\normalsize Cs. F\'is., Astro., Geof\'is. \\ 2019-1}
\chead{\bf\normalsize Depto. de F\'isica \\ Universidad de Concepci\'on}
%\rfoot{\thepage\ / pages}
\cfoot{ }
\lfoot{\tiny FB/GR}
\begin{flushleft}
\vspace{0.2in}
%\hbox to \textwidth{Nombre: \enspace \hrulefill}
%Nombre : \\
\vspace{0.25cm}
\end{flushleft}
%%%%%%%%%%%%%%%%%%%%%%%%%%%%%%%%%%%%%%%%%%

\section*{Estructuras de control}
\section{Repetitivas}
\begin{enumerate}
 \item 
Escriba un programa en Fortran 90 que permita contar en forma creciente de 1 a $N$, donde $N$ es 
ingresado por el usuario a trav\'es del teclado. Primero use la estructura repetitiva DO y despu\'es un DO WHILE.

\item 
Escriba un programa en Fortran 90 que permita contar en forma decreciente de $N$ a $-N$, 
donde $N$ es ingresado por el usuario a trav\'es del teclado.

\item 
Escriba un programa que permita calcular la suma de los primeros $N$ n\'umeros enteros, 
donde $N$ es ingresado por el usuario a trav\'es del teclado.

\item 
Escriba un programa que permita calcular la suma de los primeros $N$ n\'umeros enteros pares, donde $N$ es ingresado por el usuario a trav\'es del teclado.

\item 
Calcular el producto de los primero $N$ pares con los primeros $N$ impares.

\item 
Escriba un programa que permita calcular el factorial de un n\'umero entero.
\item 
Escriba un programa  que muestre en pantalla, los primeros $N$ elementos de la sucesi\'on $a_{n}=\dfrac{1}{2^{n}}$.

\item 
La serie de Fibonacci: 1, 1, 2, 3, 5, 8, 13,\dots se define que cada t\'ermino subsiguiente 
 est\'a dado por la suma de los dos t\'erminos anteriores: $1+1=2$;  $1+2=3$;  $2+3= 5$ \dots . Escriba un programa que muestre en pantalla los primero 100 t\'erminos de esta serie.
Modifique el programa anterior para determinar la suma de los primeros $N$ elemento de la sucesi\'on.
\item 
Escriba un programa  que permita ver las 100 primeras iteraciones del mapa
$x_{n+1}=A\,x_{n}(1-x_{n}) $  para $A= 3.3$,  $A= 3.5$   y  $A=3. 57$. Use $x_{0}= 0.5$.

\end{enumerate}
\section{Condicional}
\begin{enumerate}

 \item 

Escriba un programa que ordene de menor a mayor, una secuencia de 3 n\'umeros que son 
entrados a trav\'es  del teclado, adem\'as que calcule el promedio de ellos.
\item 
 Ingrese dos n\'umeros reales por teclado. Si $a$ es mayor que $b$ muestre la suma, en caso contrario el producto, y que si $a$ es igual a $b$ muestre la resta. 
\item 
Escriba un programa para determinar si un  entero es par o impar. 
\item 
Escriba un programa para determinar si un  entero es cuadrado perfecto. 
\item
Escriba un programa que le permita resolver la ecuaci\'on de segundo grado, $ax^{2}+bx+c=0$ donde los valores
 de $a$, $b$ y $c$ son valores reales, ingresados por el usuario a trav\'es del teclado.  Adem\'as, debe indicar 
si NO tiene ra\'ices reales e incluir el caso si $a = 0$.
\item
 Dada la sucesi\'on $S_n=80*n-n^2+200$ con $n$ variando de 1 a 100. Determinar cu\'antos elementos son 
positivos, negativos  y ceros.
\item
A partir de que valor de $n$ la sucesi\'on $a_{n}=\dfrac{1}{2n+4}$ varia menos del  2\%. 
Definiendo variaci\'on relativa como $\vert\frac{a_{n+1}-a_{n}}{a_{n}}\vert$.
\item 
Escriba un programa que le permita obtener los d\'ias de los meses del a\~no, ingresando el mes (1-12).

\item Ingresar la hora del d\'ia por teclado y el programa lo debe saludar con: buenos d\'ias, buenas tardes o buenas noches.
\end{enumerate}

\end{document}
