\documentclass[11pt]{exam}
\usepackage[activeacute,spanish]{babel} % Permite el idioma espa\~nol.
\usepackage[latin1]{inputenc}
\usepackage{amsmath,epsfig}
%\usepackage{graphdicx}
\usepackage[colorlinks]{hyperref}
\usepackage{minted} 
\usemintedstyle{emacs}
\usepackage{tcolorbox} % colores para el fondo
\definecolor{bg}{rgb}{0.95,0.95,0.95} %color de fondo

\pagestyle{headandfoot}

\begin{document}

\firstpageheadrule
%\firstpagefootrule
%\firstpagefooter{}{Pagina \thepage\ de \pages}{}
\runningheadrule
%\runningfootrule
\lhead{\bf\normalsize Computaci\'on Cient\'ifica\\ Gu\'ia 02}
\rhead{\bf\normalsize Cs. F\'is., Astro., Geof\'is. \\ 2018-1}
\chead{\bf\normalsize Depto. de F\'isica \\ Universidad de Concepci\'on}
%\rfoot{\thepage\ / pages}
\cfoot{ }
\lfoot{\tiny GR}
\begin{flushleft}
\vspace{0.2in}
%\hbox to \textwidth{Nombre: \enspace \hrulefill}
%Nombre : \\
\vspace{0.25cm}
\end{flushleft}
%%%%%%%%%%%%%%%%%%%%%%%%%%%%%%%%%%%%%%%%%%

%\begin{center}
%\texttt{Fecha de Entrega: Jueves 28 de Agosto. Env'ie a gfrubi@udec.cl los archivos .py que resuelven cada uno de los problemas propuestos.}
%\end{center}
\begin{questions}

\item En alg'un editor simple de texto (por ejemplo, \texttt{nano} en la consola, o bien \texttt{gedit} o \texttt{geany} en el ambiente gr'afico \texttt{Xfce}. Ante cualquier duda consulte al profesor encargado), genere un archivo de texto llamado \texttt{plantilla.tex}, con el siguiente contenido:

\begin{minted}[bgcolor=bg]{tex}
\documentclass[12pt]{article}

\begin{document}
\'Este es mi primer documento en \LaTeX
 
\end{document}
\end{minted}

\item Compile el c'odigo \LaTeX\ con el comando \texttt{pdflatex plantilla.tex}. Si todo sale bien, debe generar directamente un archivo .pdf con su primer trabajo en \LaTeX.
Abra el archivo .pdf para visualizar el resultado.


\item Para saciar su infinita curiosidad, mire (en la consola!) el contenido de los archivos auxiliares generados (.aux y .log). Luego de esto, borre todos los archivos generados por la compilaci'on.


\item Usando el comando \texttt{cp} haga dos copias de su archivo \texttt{plantilla.tex} con nombres \texttt{test-01.tex} y \texttt{test-02.tex}. Guarde el archivo \texttt{plantilla.tex} en alg'un lugar seguro, le servir'a en el futuro.

\item Agregue a su archivo \texttt{test-01.tex} algunas secciones y texto que involucre caracteres latinos, usando \verb|\'a, \'e, \'i, \'o, \'i, \~n y ?`|, que generan \'a, \'e, \'i, \'o, \'i, \~n, y ?`, respectivamente

\item Ahora agregue el siguiente c'odigo en alguna parte de su documento:

\begin{minted}[bgcolor=bg]{tex}
\begin{quote}
``El primer principio es que no te debes enga\~nar a ti mismo - y t\'u eres 
la persona que m\'as f\'acilmente te enga\~na. As\'i que hay que tener mucho 
cuidado con eso. Una vez que no te enga\~nas a ti mismo, es f\'acil que no 
enga\~nes a los otros cient\'ificos''. \texttt{Richard Feynman}.
\end{quote}
\end{minted}

Esto introduce el texto dentro del entorno \texttt{quote}, que es apropiado para citar frases c'elebres de alg'un personaje importante. Vea c'omo luce el resultado en su archivo .pdf.

\texttt{Ojo!} Existen tres tipos de comillas: 	las comillas ``simples'' ('\,), las comillas ``dobles'' ("\,), y las comillas ``diagonales hacia la derecha'' (`). \'Estas se obtienen con combinaciones distintas de teclas (que var'ian de teclado en teclado!). Las comillas usadas en el ejemplo del entorno \texttt{quote} son dos comillas diagonales al comienzo y dos comillas simples al final de la frase.


\item Cambie el tipo de entorno usado en el punto anterior desde \texttt{quote}, para que ahora sea un entorno \texttt{center}, \texttt{flushleft}, \texttt{flushright} y finalmente \texttt{sloppypar}. En cada caso, vea c'omo esto afecta al resultado final.

\item Lea el pdf de la \href{https://github.com/gfrubi/CC/blob/master/LaTeX/clases-LaTeX.pdf}{presentaci\'on de \LaTeX} usada en clases, hasta la p'agina 23 (``Espa\~nol y \LaTeX'').

\item Descargue el archivo modelo \href{https://github.com/gfrubi/CC/blob/master/guias/02/articulo.pdf}{articulo.pdf} (link tambi'en disponible en el \href{http://cc-cfm.blogspot.com}{blog} del curso) y 'abralo para ver qu'e contiene. 


\item Edite \texttt{test-02.tex} para que al compilarlo se reproduzca lo m'as fielmente posible el contenido del model en el archivo \texttt{articulo.pdf} (secciones, subsecciones, listas, texto, etc.).

\item En el archivo \texttt{test-02.tex} realize las siguientes modificaciones y observe qu'e efecto tiene cada una de ellas en el .pdf final.
\begin{parts}
\item Agregue el comando \verb|\tableofcontents| en la l'inea siguiente a \verb|\begin{document}|. No olvide compilar dos veces para ver el efecto de este cambio!.
\item Agregue el comando \verb|\usepackage[spanish]{babel}| en la segunda l'inea del c'odigo, es decir, en la l'inea siguiente a \verb|\documentclass[12pt]{article}|.
\item Agregue la opci'on \verb|twocolumn| a la declaraci'on de clase de la primera l'inea, es decir, transf'ormela en \verb|\documentclass[12pt,twocolumn]{article}|.
\item Finalmente, modifique la opci'on \verb|12pt| en la primera l'inea, reemplaz'andola por \verb|10pt|.
\end{parts}

\item \LaTeX es un mundo vasto, bello y desconocido, en el que se pueden seguir aprendiendo y desarrollando nuevos aspectos constantemente. Para explorar un poco m'as, descargue y d'e un vistazo al tutorial ``\textit{La introducci'on no-tan-corta a \LaTeX 2e}'' (2014), de Tobias Oetiker, Hubert Partl, Irene Hyna y Elisabeth Schlegl, disponible en \url{http://www.ctan.org/tex-archive/info/lshort/spanish}. Note que en la subcarpeta \texttt{fuente/src} del link anterior est'a disponible el c'odigo \LaTeX que genera este documento.

\item Finalmente, otra buena referencia para aprender y/o consultar sobre \LaTeX es el libro ``\textit{Edici'on de Textos Cient'ificos en \LaTeX: Composici'on, Dise\~no Editorial, Gr'aficos, Inkscape, Tikz y Presentaciones Beamer}'' (2da edici'on. Actualizaci'on Marzo 2017), de Walter Mora y Alex'ander Borb'on, disponible en \url{http://tecdigital.tec.ac.cr/revistamatematica/Libros/LATEX/}. Descargue este libro, mire qu'e contiene y gu'adelo para refencia futura.
\end{questions}
\end{document} 