\documentclass[11pt]{exam}
%\usepackage[activeacute,spanish]{babel} % Permite el idioma espa\~nol.
\usepackage[latin1]{inputenc}
\usepackage{amsmath,amsfonts}
\usepackage[colorlinks]{hyperref}
\usepackage{graphicx}
%\usepackage{minted} 

\pagestyle{headandfoot}

%%%%%%%%%%%%%%%%%%%%%%%%%%%%%%%%%%%%%%%%%%%%%
\begin{document}
\firstpageheadrule
%\firstpagefootrule
%\firstpagefooter{}{Pagina \thepage\ de \pages}{}
\runningheadrule
%\runningfootrule
\lhead{\bf\normalsize Computaci\'on Cient\'ifica\\ Gu\'ia 14}
\rhead{\bf\normalsize Cs. F\'is., Astro., Geof\'is. \\ 2018-1}
\chead{\bf\normalsize Depto. de F\'isica \\ Universidad de Concepci\'on}
%\rfoot{\thepage\ / pages}
\cfoot{ }
\lfoot{\tiny FB}
\begin{flushleft}
\vspace{0.2in}
%\hbox to \textwidth{Nombre: \enspace \hrulefill}
%Nombre : \\
\vspace{0.25cm}
\end{flushleft}
%%%%%%%%%%%%%%%%%%%%%%%%%%%%%%%%%%%%%%%%%%

\section{Funciones}
\begin{enumerate}

\item Escriba un programa que muestra los valores de las funciones intr\'insecas, seno, coseno y tangente para los \'angulos: 0, $\pi /6$, $\pi /4$, $\pi /3$, $\pi /2$, $\pi$, $2\pi/3$, $3\pi /2$
y $2 \pi$.
\item Escriba una funci\'on que permita evaluar el seno hiperb\'olico de una variable real, donde $senh(x)=\dfrac{\exp(x)-\exp(-x)} {2}$, cuando
el usuario ingresa el valor de $x$ a trav\'ez del teclado en el programa principal.
\item Cree un programa que analiza n\'umeros enteros que contenga las siguentes funciones:
\begin{itemize}
 \item que indique si el n\'umero es par o impar, 
\item  que indique si el n\'umero es positivo o negativo

\end{itemize}

\item Usando function escriba un programa que evalue la siguiente  funci\'on real, donde $x\in \Re$:
 $$f(x)=\left\lbrace 
\begin{array}{ll}
x+ 2 &  x \geq 0 \\
-x +2 & x < 0 \\
\end{array}\right. $$
\item 
Desarrollar un algoritmo de c\'alculo que eval\'ue una funci\'on $f(x,y)$ para diferentes
valores de $x$ e $ y$. La funci\'on est\'a definida de la siguiente manera:
$$f(x,y)=\left\lbrace 
\begin{array}{lll}
x+ y &  x \geq 0 & \wedge \quad  y \geq 0\\
x + y^{2} & x \geq 0 & \wedge \quad y < 0\\
x +y & x < 0 & \wedge \quad  y\geq0\\
x^{2}  + y^{2} &  x < 0 & \wedge \quad y < 0\\
\end{array}\right. $$
  El algoritmo de c\'alculo debe desarrollarse con las estructuras de control necesarias para
que realice los siguientes pasos:
\begin{itemize}
\item  Solicite al usuario los valores de: $x$ e $y$
\item Eval\'ue la funciones seg\'un corresponda
\item Despliegue el valor de la funci\'on resultante
\end{itemize}

\end{enumerate}
\section{Subrutinas}
\begin{enumerate}
\item Escriba una subrutina que  entrege la ecuaci\'on de una l\'inea recta, cuando
el usuario ingresa dos puntos por teclado en el programa principal.
 \item Escriba una subrutina que resuelve la ecuaci\'on de segundo grado.
\item  Cree una subrutina que entrege el promedio aritm\'etico de $N$ notas de un alumno.
\item Cree una subrutina que encuentra el valor m\'aximo entre dos n\'umeros reales.
\end{enumerate}
%
\section{Funciones intr\'insecas b\'asicas}
\begin{enumerate}
 \item Escriba un programa donde se usen todas las funciones que est\'an el la siguiente tabla\\


\begin{tabular}{|p{2.5cm}|p{6.5cm}|p{3.5cm}|p{3.5cm}|}
\hline 
\textbf{Funci\'on} &         \textbf{Significado matem\'atico} &                 \textbf{Tipo argumento}& \textbf{Tipo resultado}\\ \hline 

INT(x)&           Parte entera de x &                      REAL &          INTEGER\\ \hline 
FLOOR(x)&         Mayor entero $\leq$x  &                      REAL &          INTEGER\\ \hline 
MOD(x, y)&        Resto de la divisi\'on: x-INT(x/y) &       INTEGER \'o REAL& Como el argumento\\ \hline 
NINT(x)  &        x redondeado al entero m\'as pr\'oximo&      REAL  &         INTEGER\\ \hline 
REAL(x)  &        Convierte x a REAL  &                 INTEGER&        REAL\\ \hline 
MAX(x1,$\cdots$,xn)&     M\'aximo de x1, x2, $\cdots$,xn &                  INTEGER \'o REAL& Como el argumento\\ \hline 
MIN(x1,$\cdots$,xn) &    M\'inimo de x1, x2, $\cdots$,xn &                  INTEGER \'o REAL& Como el argumento\\ \hline 
ABS(x)   &        Valor absoluto de x    &                 INTEGER \'o REAL& Como el argumento\\ \hline 
LOG(x)   &        Logaritmo natural de x &                 REAL           &REAL\\ \hline
LOG10(x)   &        Logaritmo base 10 de x &                 REAL           &REAL\\ \hline  
EXP(x)   &        Funci\'on exponencial    &                 REAL           &REAL\\ \hline 
COS(x)   &        Coseno de x (en radianes)&               REAL           &REAL\\ \hline 
SIN(x)   &        Seno de x (en radianes)   &              REAL           &REAL\\ \hline 
TAN(x)   &        Tangente de x (en radianes)&             REAL           &REAL\\ \hline 
ACOS(x)  &        Arco coseno de x           &             REAL           &REAL\\ \hline 
ASIN(x)  &        Arco seno de x             &            REAL           &REAL\\ \hline 
ATAN(x)  &        Arco tangente de x         &             REAL           &REAL\\ \hline 
SQRT(x)  &        Ra\'iz cuadrada de x &                     REAL           &REAL\\ \hline 

\end{tabular} 
\end{enumerate}
%

\end{document}