\documentclass[11pt]{exam}
\usepackage[activeacute,spanish]{babel} % Permite el idioma espa\~nol.
\usepackage[utf8]{inputenc}
\usepackage{amsmath,epsfig}
%\usepackage{graphdicx}
\usepackage[colorlinks]{hyperref}
\usepackage{minted} 
\usemintedstyle{emacs}
\usepackage{tcolorbox} % colores para el fondo
\definecolor{bg}{rgb}{0.95,0.95,0.95} %color de fondo

\pagestyle{headandfoot}

\begin{document}


\firstpageheadrule
%\firstpagefootrule
%\firstpagefooter{}{Pagina \thepage\ de \pages}{}
\runningheadrule
%\runningfootrule
\lhead{\bf\normalsize Computaci\'on Cient\'ifica\\ Gu\'ia 01}
\rhead{\bf\normalsize Cs. F\'is., Astro., Geof\'is. \\ 2019-1}
\chead{\bf\normalsize Depto. de F\'isica \\ Universidad de Concepci\'on}
%\rfoot{\thepage\ / pages}
\cfoot{ }
\lfoot{\tiny GR}
\begin{flushleft}
\vspace{0.2in}
%\hbox to \textwidth{Nombre: \enspace \hrulefill}
%Nombre : \\
\vspace{0.25cm}
\end{flushleft}
%%%%%%%%%%%%%%%%%%%%%%%%%%%%%%%%%%%%%%%%%%

\begin{questions}

\item Cada computador de los laboratorios Linux de nuestra Facultad est'a configurado con \textit{una cuenta com'un}, es decir, a la que potencialmente \textit{acceden todos los estudiantes} que hacen uso de ellos. Por esto, recuerde que \textbf{usted es responsable de traer los archivos que requiera para su trabajo, as'i como de respaldar su trabajo al final de cada sesi'on}.

\item Antes de comenzar a trabajar la persona a cargo les mostrar\'a brevemente
el escritorio (\href{http://www.xfce.org/?lang=es}{Xfce}) instalado en el sistema (Ubuntu, y en particular la variante \href{http://xubuntu.org/}{Xubuntu}) y las aplicaciones b\'asicas: navegador de archivos, editor de texto, consola virtual, etc.

\item Busque y abra la consola (virtual) de comandos o \textit{terminal}. En esta consola realice las siguientes tareas:
\begin{parts}
\item Liste el contenido de su carpeta, ejecutando el comando 

\begin{minted}[bgcolor=bg]{bash}
ls
\end{minted}

Verifique que los archivos listados son los que est'an en la carpeta de usuario.

\item Cree el directorio o carpeta \textbf{test1}, ejecutando el comando 

\begin{minted}[bgcolor=bg]{bash}
mkdir test1
\end{minted}

Verifique la creaci'on de la carpeta usando el navegador de archivos.

\item Ingrese a la carpeta reci'en creada, ejecutando el comando 

\begin{minted}[bgcolor=bg]{bash}
cd test1
\end{minted}

y en su interior cree una nueva carpeta llamada \textbf{test2}.

\item Usando un editor de texto cree un archivo ascii/UTF8 (es decir, de texto simple) con el nombre \textbf{archivo1.txt}, e ingrese como contenido su nombre completo.

\item Edite ahora el mismo archivo \textit{en la consola}, con el editor ``nano'', ejecutando el comando 

\begin{minted}[bgcolor=bg]{bash}
nano archivo1.txt
\end{minted}

 Explore las opciones de este editor de texto. ?`C'omo se guardan los archivos creados/modi\-fi\-ca\-dos?. ?`C'omo se sale del editor?.

\item Mueva el archivo \textbf{archivo1.txt} a \textbf{archivo2.txt}, ejecutando el comando

\begin{minted}[bgcolor=bg]{bash}
mv archivo1.txt archivo2.txt
\end{minted}

En este caso, el efecto es equivalente a haber renombrado el archivo.

\item Cambie el nombre de la carpeta \textbf{test2} a \textbf{temp}.

\end{parts}

\item Conozca el comando \texttt{cat}, que con\textbf{cat}ena (es decir, une) archivos y los despliega en la pantalla (la salida est'andar).
\begin{parts}
\item  Inf'ormese de los aspectos generales de este comando desplegando el manual, con 

\begin{minted}[bgcolor=bg]{bash}
man cat
\end{minted}



\item En la consola, cree dos archivos \textbf{test01.txt} y \textbf{test02.txt} usando el editor \texttt{nano} e introduzca en ellos alg'un texto interesante.

\item Ejecute ahora el comando 

\begin{minted}[bgcolor=bg]{bash}
cat test01.txt test02.txt
\end{minted}

Observe c'omo los dos archivos se despliegan en pantalla, uno luego del otro.

\item ?`Cu'al es la diferencia del comando anterior respecto a 

\begin{minted}[bgcolor=bg]{bash}
cat *.txt
\end{minted}

Compruebe su respuesta con un ejemplo.
\end{parts}

\item Descargue el archivo \href{https://github.com/gfrubi/CC/raw/master/guias/01/Quijote.tar.gz}{Quijote.tar.gz}, disponible tambi'en en el \href{http://cc-cfm.blogspot.com}{blog del curso}. Este archivo contiene (en forma comprimida) cinco archivos correspondientes a algunas secciones iniciales de la obra ``Don Quijote de la Mancha"\, de Miguel de Cervantes. Abra una consola e ingrese (usando el comando \texttt{cd}) a la carpeta donde se encuentra el archivo descargado y realice las siguientes tareas:
\begin{parts}
\item Descomprima el archivo, usando el comando 

\begin{minted}[bgcolor=bg]{bash}
tar -xf Quijote.tar.gz
\end{minted}

Esto crear'a cinco archivos con extensi'on \textbf{.txt}. 
\item Inf'ormese de los aspectos generales del comando \texttt{tar}, desplegando el manual del comando \texttt{tar} (ingrese \texttt{man tar}).
\item Despliege el contenido de cada archivo usando el comando \texttt{more}.
\item Una los archivos (en orden!) y guarde el resultado en el archivo \textbf{Quijote.txt}. Para esto, la opci'on \texttt{>} para redireccionar la salida de \texttt{cat} al archivo correspondiente.
\end{parts}

\item Conozca el comando \texttt{grep}, que busca (conjuntos de) palabras dentro de archivos de texto. 
\begin{parts}
\item Busque informaci'on en internet sobre este comando. Por ejemplo, realice la b'usqueda ``comando grep linux'' en Google. 'Este es un m'etodo que usualmente permite encontrar r'apidamente informaci'on 'util. Por ejemplo, el correspondiente \href{https://es.wikipedia.org/wiki/Grep}{art\'iculo de Wikipedia} es un buen punto de partida.

\item Usando \texttt{grep}, busque todas las ocurrencias de la palabra \textbf{Mancha} en cada uno de los cinco archivos originales del Quijote (\textbf{c1.txt} \dots \textbf{c5.txt})

\item Agregue la opci'on \texttt{-n} al comando \texttt{grep} usado anteriormente (es decir, use \texttt{grep -n}). ?`Qu'e efecto tiene sobre el resultado?
\end{parts}

\item Puede escribir y ejecutar una secuencia de comandos almacenados en un ``\textit{script}'' (archivo de comandos).
\begin{parts}
\item Para este ejemplo, cree una nueva carpeta y cree en ella una copia del archivo comprimido \textbf{Quijote.tar.gz}.

\item En esta carpeta, cree un archivo con nombre \textbf{mi-script.sh} (la extensi'on .sh es opcional, pero conveniente). En este archivo escriba una variaci'on de los  comandos antes usados, cada uno en una l'inea separada, que realicen las siguientes acciones: 1) descomprima el archivo \textbf{Quijote.tar.gz}, 2) Una los archivos .txt y crea el archivo \textbf{Quijote.txt} y 3) busca todas las ocurrencias de la palabra \textbf{batalla} en este archivo y 4) crea el archivo \textbf{resultado.txt} con el resultado de la b'usqueda.

\item Ejecute su script (es decir, los comandos que contiene el archivo \textbf{mi-script.sh} en el orden en que est'an escritos), ejecutando 

\begin{minted}[bgcolor=bg]{bash}
bash mi-script.sh
\end{minted}
\end{parts}

\item Descargando y manipulando datos reales.

\begin{parts}
\item Desde el sitio del \textit{Supernova Cosmology Project} (un proyecto de colaboraci'on cient'ifica internacional que reune datos y estudios sobre explosiones de supernovas en el Universo, \url{http://supernova.lbl.gov/}), descargue el archivo de datos  \url{http://supernova.lbl.gov/Union/figures/SCPUnion2.1_mu_vs_z.txt}. En lugar de descargar este archivo usando el navegador, como es usual, h'agalo directamente desde la consola usando el comando \texttt{wget}. En particular, ejecute 

\begin{minted}[bgcolor=bg]{bash}
wget http://supernova.lbl.gov/Union/figures/SCPUnion2.1_mu_vs_z.txt
\end{minted}

\item El archivo descargado contiene datos reales correspondientes a la observaci'on de explosiones de Supernovas. Despliege el archivo usando el comando \texttt{more}. La primera columna lista el nombre de la supernova, que comienza con el a\~no en que fue observada. Otras columnas contienen datos f'isicos de relevancia como el m'odulo de distancia y el corrimiento hacia el rojo (redshift) medido para cada supernova. M'as adelante, en este curso, graficaremos estos datos (paciencia!). Note que el listado NO est'a ordenado por a\~no. 

\item Usando comandos bash, encuentre todas las supernovas que fueron observadas el a\~no 2004, y almacene sus datos en un nuevo archivo, llamado \textbf{2004.txt}.
\end{parts}


\item Puede usar el tiempo restante para aprender sobre otros comandos Bash. Algunos de ellos est'an listados \href{https://es.wikipedia.org/wiki/Comandos_Bash}{este art{\'\i}culo de Wikipedia}.


\end{questions}
\end{document} 