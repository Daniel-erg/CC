\documentclass[11pt]{exam}
\usepackage[activeacute,spanish]{babel} % Permite el idioma espa\~nol.
\usepackage[utf8]{inputenc}
\usepackage{amsmath,epsfig}
\usepackage[colorlinks]{hyperref}

\usepackage{minted} 
\usemintedstyle{emacs}
\usepackage{tcolorbox} % colores para el fondo
\definecolor{bg}{rgb}{0.95,0.95,0.95} %color de fondo

\pagestyle{headandfoot}
\spanishdecimal{.}

\begin{document}

\firstpageheadrule
\runningheadrule
\lhead{\bf\normalsize Computaci\'on Cient\'ifica\\ Gu\'ia 10}
\rhead{\bf\normalsize Cs. F\'is., Astro., Geof\'is. \\ 2019-1}
\chead{\bf\normalsize Depto. de F\'isica \\ Universidad de Concepci\'on}
\cfoot{ }
\lfoot{\tiny FB/GR}
\begin{flushleft}
\vspace{0.2in}

\vspace{0.25cm}
\end{flushleft}
%%%%%%%%%%%%%%%%%%%%%%%%%%%%%%%%%%%%%%%%%%

\section{Parte I, construcci\'on de arrays}
Para  construir un array de dimensi\'on 1 (vector) que contiene valores espec\'ificos, los valores se pueden entregar list\'andolos, escribiendos uno a uno, o bien usando un bucle impl\'icito o una combinaci\'on de ambos:
\begin{center}
\begin{tabular}{llll}
A = &(/ 1, 2, 3, 4, 5,6/) 		&! = (/1,2,3,4,5,6/) &¸\\
B = &(/(i, i = 1,7)/)         		&! = (/1,2,3,4,5,6,7/)&\\
C = &(/7, (i, i=1,4), 9/)     		&! = (/7,1,2,3,4,9/) & \\
D = &(/(i**2, i = 1,6)/)		&! =(/1, 4, 9, 16, 25, 36/) & \\
E = &(/((i+j, i=1,3), j=1,2)/)		&! = (/((1+j,2+j,3+j),j=1,2)/)	&! = (/2,3,4,3,4,5/) 
\end{tabular} 
\end{center}
\begin{questions}
 \item 
Escriba un programa en  fortran 90 que muestre en pantalla los valores de los vectores de A hasta E.
\item 
Escriba un programa en  fortran 90 que sume todos los vectores compatible del problema anterior y muestre el resultado en pantalla.
\item 
Funciones intr\'insecas para arrays. Aplicar las siguientes funciones a los vectores A hasta E:  a) \texttt{SUM(array)} b) \texttt{MAXVAL(array)} c) \texttt{MINVAL(array)} d) \texttt{PRODUCT(array)}. Discuta el resultado.

Para  construir un array de rango 2 (matriz) conteniendo valores especificados. Los valores se pueden dar list\'andolos o bien usando un bucle impl\'icito o una combinaci\'on de ambos:\\
Ejemplo  integer F(3,3), G(2,2)

\begin{tabular}{llll}
F(1,1:3)& =& (/1, 3, 5/)    	&! ingresados por filas\\
F(2,1:3)& = &(/2, 6, 10/) 	&! ingresados por filas\\
F(3,1:3)& = &(/3, 9, 15/)	&! ingresados por filas\\
G(1:2,1)& = &(/(2*i, i = 1,2)/)  &! ingresados por columnas\\
G(1:2,2)& = &(/(3*i+2, i = 1,2)/)&! ingresados por columnas\\
\end{tabular}
\item 
Escriba un programa que muestre en pantalla los valores de las matrices F y G
\item 
Escriba un programa que muestre en pantalla los valores de la suma de cada columna de las matrices F y G. Nota: usar \texttt{SUM(array, DIM=1)}.
\item 
Escriba un programa que muestre en pantalla los valores de la suma de cada fila de las matrices F y G. Nota, usar \texttt{SUM(array, DIM=2)}.
\item 
Escriba un programa que muestre en pantalla los m\'aximos valores de las matrices F y G, trabaje por columnas y filas. Nota, usar \texttt{MAXVAL(array, DIM=1)} y \texttt{MAXVAL(array, DIM=2)} repectivamente.
\item 
Dada una matriz A, explorar las operaciones aritm\'eticas para matrices de fortran 90, es decir, \texttt{A*A, A/A, A+A y A-A}, discuta el resultado.
\item 
Escriba un programa     que permita calcular el valor promedio de los elementos que pertenecen a un arreglo bidimensional A, de tipo real, utilizando la ecuaci\'on:
$$prom=\dfrac{1}{m*n}\sum a_{ij}$$
\\

\noindent
\section{Parte II, arrays din\'amicos y aplicaciones}
Fortran 90 permite obtener y liberar memoria din\'amica a trav\'es de los arrays din\'amicos. Permite a los arrays locales a un procedimiento tener tama\~nos y formas diferentes en cada llamada a trav\'es de los vectores autom\'aticos. Reduce los recursos globales necesarios para almacenamiento en memoria.
Ejemplo\\
\begin{minted}[bgcolor=bg]{fortran}
Program array_dinamicos
   integer :: N
   real, dimension(:,:), allocatable :: A
   print*,"ingrese la extension de la matriz cuadrada"
   read*, N
   allocate(A(N,N))
   do j=1,N
     A(j,1:N)=(/(i, i=1,N)/)
     print*,A(j,1:N)
   end do
   deallocate(A)
end program 
\end{minted}
\item 
Escriba un programa, que ordene de mayor a menor 4 n\'umeros enteros que fueron ingresados por el(la) usuario(a).
\item 
Escriba un programa que permita ingresar un vector (arreglo unidimensional) de tipo entero desde el teclado y luego emitirlo por pantalla. Suponer: a) el n\'umero de elementos del arreglo es conocido; b) el n\'umero de elementos del arreglo no es conocido.  
\item 
Escriba un programa que permita calcular el promedio de un vector de tipo real de tama\~no $N$, previamente ingresado por teclado. 
\item 
Escriba un programa que permita calcular el producto escalar de dos vectores de tipo real de tama\~no $N$, compare su resultado con la funci\'on intr\'inseca  \texttt{DOT\_PRODUCT(arg1,arg2)}.
\item     
Escriba un programa que permita evaluar un polinomio $P_{n}(x)$ de la forma:
\begin{equation}
P_{n}(x)=a_{n}x^{n}+a_{n-1}x^{n-1}+\ldots+a_{1}x^{1}+a_{0}x^{0},
\end{equation}
suponiendo que los coeficientes $a_{n}$ son reales y  el entero $n$ son datos entregados por el(la) usuario(a).
\item 
Escriba un programa  que permita determinar el mayor de un arreglo de $N$ n\'umeros enteros ingresados por teclado.
\item 
Modifique el programa anterior para que determine el mayor y el menor de un arreglo de $N$ n\'umeros enteros.
\item 
Escriba un programa  que permita ingresar un arreglo bidimensional de tipo real desde el teclado y luego emitirlo por pantalla. Suponer que se conoce el tama\~no del arreglo.

\item 
Escriba un programa que permita obtener la suma de los elementos de la diagonal principal y la suma de los elementos de la diagonal secundaria de una matriz cuadrada. 
\begin{equation}
A=\begin{bmatrix}
 1&2&3\\
4&5&6\\
7&8&9\\
\end{bmatrix}
\end{equation}
Ejemplo: En la matriz el vector $[1, 5, 9]$ es el que se obtiene a partir de la diagonal principal, y el vector $[3, 5, 7]$ el que se obtiene a partir de la diagonal secundaria. 

\item 
Escriba un programa  que permita hacer la suma de dos matrices $A$ y $B$, suponer que las matrices son conocidas y compatibles.
\item 
Sea $A$ una matriz conocida, escriba un programa que obtenga la matriz resultante de la multiplicaci\'on de un escalar $K$ por $A$.
\item 
Dada una matriz $A(n,m)$ escriba un programa que obtenga la matriz transpuesta.
\item 
Escriba un programa  que permita hacer el producto de dos matrices $A$ y $B$, suponer que las matrices son conocidas. Comparar su resultado con la funci\'on \texttt{MATMUL(Arg1,Arg2)}.
\item 
Dada una matriz de $2\times 2$, obtenga el determinante.


\end{questions}
%
\end{document}
