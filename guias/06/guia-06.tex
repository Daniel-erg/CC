\documentclass[11pt]{exam}
\usepackage[activeacute,spanish]{babel} % Permite el idioma espa\~nol.
\usepackage[utf8]{inputenc}
\usepackage{amsmath,epsfig}
\usepackage[colorlinks]{hyperref}

\usepackage{minted} 
\usemintedstyle{emacs}
\usepackage{tcolorbox} % colores para el fondo
\definecolor{bg}{rgb}{0.95,0.95,0.95} %color de fondo

\pagestyle{headandfoot}

\begin{document}

\firstpageheadrule
\runningheadrule
\lhead{\bf\normalsize Computaci\'on Cient\'ifica\\ Gu\'ia 06}
\rhead{\bf\normalsize Cs. F\'is., Astro., Geof\'is. \\ 2018-1}
\chead{\bf\normalsize Depto. de F\'isica \\ Universidad de Concepci\'on}
\cfoot{ }
\lfoot{\tiny GR}
\begin{flushleft}
\vspace{0.2in}

\vspace{0.25cm}
\end{flushleft}
%%%%%%%%%%%%%%%%%%%%%%%%%%%%%%%%%%%%%%%%%%

\begin{questions}

\item Aprenda sobre la clase \textbf{Beamer} de \LaTeX, diseñada para producir archivos .pdf para ser usados en presentaciones. Puede leer sobre esta clase en \href{https://es.wikipedia.org/wiki/Beamer}{esta página} en Wikipedia. Una presentación sobre Beamer (y escrita en Beamer) está disponible en español \href{http://metodos.fam.cie.uva.es/~latex/apuntes/apuntes13.pdf}{aqu\'i}.

\item En Beamer, la unidad estructural básica es un ``frame'' (que corresponde a una ``transparencia"). Éstas son definidas con el entorno \texttt{frame}. Además, es  común definir ``bloques'' con los entornos \texttt{block} o bien \texttt{alertblock}. Para aprender el uso b'asico de Beamer, descargue el archivo de ejemplo \href{https://github.com/gfrubi/CC/blob/master/guias/06/ejemplo-beamer.tex}{\texttt{ejemplo-beamer.tex}} desde el repositorio del curso, mire su contenido y compílelo (con \texttt{pdflatex}). 

\item Modifique el archivo ejemplo, cambiando primero las opciones que determinan el aspecto del pdf creado. Para esto, puede modificar las opciones de los comandos \texttt{usetheme} (tema) y \texttt{usecolortheme} (colores). Las opciones disponibles por defecto para el comando \texttt{usetheme} son: \textit{AnnArbor, Antibes, Bergen, Berkeley, Berlin, Boadilla, CambridgeUS, Copenhagen, Darmstadt,  Dresden, Frankfurt, Goettingen, Hannover, Ilmenau, JuanLesPins, Luebeck, Madrid, Malmoe, Marburg, Montpellier, PaloAlto, Pittsburgh, Rochester, Singapore, Szeged, Warsaw}. Similarmente, para \texttt{usecolortheme} existen las opciones: \textit{default, albatross, beaver, beetle, crane, dolphin, dove, fly, lily, orchid, rose, seagull, seahorse, whale, wolverine}. Puede ver el efecto de cada una de estas opciones en la ``matriz de temas"\, disponible en \href{https://hartwork.org/beamer-theme-matrix/}{esta p\'agina}.

\item Habiendo elegido un tema y esquema de colores de su gusto, modifique el archivo ejemplo para adaptar el código que escribió la semana pasada \href{https://github.com/gfrubi/CC/blob/master/guias/05/ejemplo-articulo.pdf}{``Escribiendo mi primer artículo con formato Científico, en \LaTeX''}, para ahora crear una presentación con el mismo contenido. Envíe el pdf resultante al email del Profesor. 

\item Adem'as de presentaciones, es común que se requiera crear ``posters"\, (típicamente, para ser presentados en congresos científicos). De entre las múltiples posibilidades para crear póster en \LaTeX, hoy aprenderá algo acerca de la clase \texttt{a0poster}. Puede encontrar información básica y un ejemplo en \href{https://www.latextemplates.com/template/a0poster-portrait-poster}{esta página}.

\item Como primer paso, descargue el archivo ejemplo \href{https://github.com/gfrubi/CC/blob/master/guias/06/ejemplo-poster.tex}{\texttt{ejemplo-poster.tex}} desde el repositorio del curso y estudie el código que contiene.

\item Modifique/incluya el código de su ``primer artículo"\, para ahora crear un póster, con el mismo contenido. Envíe su creación (archivo pdf) por email al profesor.

\item \textbf{Opcional}: Si quiere divertirse un rato, experimente con el paquete \href{http://hanno-rein.de/archives/349}{``Coffee Stains"}, que permite incorporar ``manchas de café"\, a su documento \LaTeX\ :). 

\item Si por alguna razón le sobra tiempo, prepárese para la próxima clase, leyendo una introducción al uso de lenguajes de programación en Ciencias, en el \href{https://github.com/gfrubi/CC/blob/master/Python/00-Computacion-Cientifica-con-Python.ipynb}{primero} de los \href{https://github.com/gfrubi/CC/tree/master/Python}{documentos relacionados con Python} contenidos en el repositorio.
\end{questions}

\end{document} 