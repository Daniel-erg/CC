\documentclass[11pt]{exam}
\usepackage[activeacute,spanish]{babel} % Permite el idioma espa\~nol.
\usepackage[utf8]{inputenc}
\usepackage{amsmath,epsfig}
\usepackage[colorlinks]{hyperref}

\usepackage{minted} 
\usemintedstyle{emacs}
\usepackage{tcolorbox} % colores para el fondo
\definecolor{bg}{rgb}{0.95,0.95,0.95} %color de fondo

\pagestyle{headandfoot}

\begin{document}

\firstpageheadrule
%\firstpagefootrule
%\firstpagefooter{}{Pagina \thepage\ de \pages}{}
\runningheadrule
%\runningfootrule
\lhead{\bf\normalsize Computaci\'on Cient\'ifica\\ Gu\'ia 06}
\rhead{\bf\normalsize Cs. F\'is., Astro., Geof\'is. \\ 2019-1}
\chead{\bf\normalsize Depto. de F\'isica \\ Universidad de Concepci\'on}
%\rfoot{\thepage\ / pages}
\cfoot{ }
\lfoot{\tiny GR}
\begin{flushleft}
\vspace{0.2in}
%\hbox to \textwidth{Nombre: \enspace \hrulefill}
%Nombre : \\
\vspace{0.25cm}
\end{flushleft}
%%%%%%%%%%%%%%%%%%%%%%%%%%%%%%%%%%%%%%%%%%

\begin{questions}

\item En alg'un archivo .tex que ya tenga hecho, inserte el escudo de la UdeC, contenido en el archivo \url{http://www.udec.cl/normasgraficas/sites/default/files/marcaderecha.png}:
\begin{figure}[h!]
\begin{center}
\includegraphics[width=4cm]{marcaderecha.png}
\end{center}
\caption{Marca alineaci'on derecha, Formato PNG.}
\label{fig:escudo}
\end{figure}

Vea \href{http://www.udec.cl/normasgraficas/node/4}{esta} p'agina para conocer otras variaciones oficiales del escudo de la UdeC. 
Puede encontrar m'as informaci'on acerca de las normas gr'aficas de nuestra Universidad en \href{http://www.udec.cl/normasgraficas}{esta} p'agina.

\item Ahora su misi'on es escribir su primer ``art'iculo cient'ifico'' en \LaTeX, que debe reproducir lo m'as fielmente posible \href{https://github.com/gfrubi/CC/blob/master/guias/06/ejemplo-articulo.pdf}{este} ejemplo. Para eso, use todo lo aprendido (en particular, use referencias cruzadas a las ecuaciones, tablas, figuras y referencias). El archivo .pdf de la figura lo puede descargar desde \href{https://github.com/gfrubi/CC/blob/master/guias/05/fig-ajuste-lineal.pdf}{aqu\'i}.

El t'itulo, autor, y resumen puede ser incorporado usando los siguientes comandos despu'es del conocido \verb|\documentclass|:

\begin{minted}[bgcolor=bg]{tex}
\title{Escribiendo mi primer art\'iculo con formato Cient\'ifico, en \LaTeX}
\author{Su nombre (autor(a))}

\begin{document}

\maketitle
\begin{abstract}
Este es el resumen del art\'iculo...
\end{abstract}
\end{minted}


\end{questions}

\end{document} 