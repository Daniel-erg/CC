\documentclass[11pt]{exam}
\usepackage[activeacute,spanish]{babel} % Permite el idioma espa\~nol.
\usepackage[utf8]{inputenc}
\usepackage{amsmath,epsfig}
\usepackage[colorlinks]{hyperref}

\usepackage{minted} 
\usemintedstyle{emacs}
\usepackage{tcolorbox} % colores para el fondo
\definecolor{bg}{rgb}{0.95,0.95,0.95} %color de fondo

\pagestyle{headandfoot}

\begin{document}

\firstpageheadrule
%\firstpagefootrule
%\firstpagefooter{}{Pagina \thepage\ de \pages}{}
\runningheadrule
%\runningfootrule
\lhead{\bf\normalsize Computaci\'on Cient\'ifica\\ Gu\'ia 07}
\rhead{\bf\normalsize Cs. F\'is., Astro., Geof\'is. \\ 2019-1}
\chead{\bf\normalsize Depto. de F\'isica \\ Universidad de Concepci\'on}
%\rfoot{\thepage\ / pages}
\cfoot{ }
\lfoot{\tiny FB/GR}
\begin{flushleft}
\vspace{0.2in}
%\hbox to \textwidth{Nombre: \enspace \hrulefill}
%Nombre : \\
\vspace{0.25cm}
\end{flushleft}
%%%%%%%%%%%%%%%%%%%%%%%%%%%%%%%%%%%%%%%%%%


\section{Fortran}
 

Primera gu\'ia de fortran. 
A continuaci\'on se mostrar\'a un ejemplo de un programa simple en fortran. Elija un editor de texto
y escriba el siguiente programa, donde el s\'imbolo ! indica un comentario este no es ejecutable. Guarde su documento como ejemplo.f90\\

\begin{minted}[bgcolor=bg]{fortran}
program ejemplo
character*8 :: m  	!declarar m como variable de tipo caracter
print*, 'ingresar un nombre'  !Comando print para mostrar en pantalla un mensaje
read*, m	  !comando read para leer la variable, m, incresada del teclado
print*,'HOLA  ', m
end program
\end{minted}

Usaremos el compilador gfortran (gnu fortran)  para compilar los programas. Siga el ejemplo\\
\begin{minted}[bgcolor=bg]{bash}
> gfortan ejemplo.f90  -o salida
\end{minted}
(compila el archivo ejemplo.f90 y construye el ejecutable salida)
\begin{minted}[bgcolor=bg]{bash}
> ./salida
\end{minted}
(ejecuta (corre) el archivo salida)

\large{Ejercicios}
\begin{enumerate}
 \item 
Sean los \underline{enteros} I=2, J=3, K=4 y L=5 calcule las siguientes expresiones FORTRAN usando la calculadora y  computadora. Compare los resultados:\\
\begin{tabular}{clclclcl}
      a)& I*L/K**J & b)&  I**K/L*J &c)&  J/I*I \\
      d)&  K**J/I/J/L &e)& K**J**J  &f)&   (I/J)*J   \\
\end{tabular}
\begin{minipage}{5cm}
\begin{tabular}{|p{3cm}|p{3cm}|}
\hline 
 Fortran& Calculadora\\ \hline \hline
 &\\ \hline
 &\\ \hline
 &\\ \hline
&\\ \hline
 &\\ \hline
&\\ \hline
\end{tabular}

\end{minipage}
\par

Conclusi\'on: \dotfill \\
..\dotfill

 
 \item 
Dado los \underline{enteros} I=2, J=3 y los \underline{reales} A=4.2 y B=2.0, evaluar las siguientes 
expresiones manualmente y usando un programa FORTRAN. Use la tabla para comparar los resultados.\\
\begin{minipage}{5cm}
\vspace{0.5cm}
   \begin{enumerate}
	\item  J/I*A
	\item  A*J/I
	\item J**I+A**B
	\item J**I+A**I
    \end{enumerate}
\end{minipage}
\begin{minipage}{5cm}
\begin{tabular}{|p{3cm}|p{3cm}|}
\hline 
 Fortran& Calculadora\\ \hline \hline
 &\\ \hline
 &\\ \hline
 &\\ \hline
&\\ \hline
\end{tabular}

\end{minipage}
\\

Conclusi\'on: \dotfill \\
..\dotfill
\item 
Escriba un programa que tome un  n\'umero positivo que llamaremos de radio  y calcule el \'area del c\'irculo.
%Escriba un programa que tome un entero positivo de tres d\'igitos, invierta el orden de los d\'igitos y escriba el entero resultante. Por ejemplo, 123 debe ser transformado en 321.
\item Escriba un programa, donde entran por teclado dos n\'umeros enteros y se muestra en pantalla la suma
 y el producto entre ambos.
\item 
Escriba un programa que le permita evaluar las  funciones $\sin(x)$, $\cos(x)$, $\tan(x)$ 
donde el valor $x$ se ingresa por teclado. Recordar que en fortran el argumento de las funciones 
trigonom\'etricas est\'a en radianes.  As\'i que, primero debe transformar  $x$ de grados a radianes 
\item 
Escriba un programa que le permita usar el teorema del coseno. El programa debe leer los dos catetos y el
 \'angulo opuesto al cateto buscado, finalmente el programa muestra en pantalla el valor del cateto buscado.
\item Escriba un programa, que le permita evaluar la funci\'on $f(x)= x^n$, con $x$ real y $n$ entero ingresados  por teclado.
\item Escriba un programa, donde entran por teclado dos pares ordenados llamados puntos y calcule y muestre en pantalla
 la ecuaci\'on de la recta.
%Tabule la funci\'on $y = x^{5}$ para x = 1, 2, 3,...10 usando (X*X*X*X*X), X**5 y X**5.0. Compare los resultados.
\item 
Escriba un programa, donde entran por teclado 10 notas de alumnos y obtener  el promedio.
\item Escriba un programa, donde entran por teclado las componentes cartesianas de un vector en 
dos dimensiones y muestre en pantallas las componentes polares.
\item 
Escriba un programa, donde entran por teclado las componentes cartesianas de dos vectores y
en pantalla debe a parecer el \'angulo entre los vectores. (Recordar producto escalar 
$\vec A \cdot \vec B=A_x B_x+A_y B_y+A_z B_z= A\, B\, \cos(\alpha)$)
\item 
Dado dos vectores en coordenadas cartesianas determinar el vector desplazamiento.
%Escriba un programa para determinar si un n\'umero entero es primo.
\item 
Escriba un programa que lea dos n\'umeros reales positivos $a$ y $b$,  que ser\'an los catetos de un
tri\'angulo rect\'angulo y muestre en pantalla la hipotenusa y el \'angulo
%\item Ingrese los 4 elementos reales de una matriz de 2x2 y calcule su determinante.
%Escriba un programa que indique para que n\'umero entero de iteraciones (n) la sucesi\'on
%$a_{n}=\left( 1+\dfrac{1}{n}\right)^{n} $ varia en 1\%.
\end{enumerate}
%
\end{document}
